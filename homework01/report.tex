%%
%% This is file `sample-acmtog.tex',
%% generated with the docstrip utility.
%%
%% The original source files were:
%%
%% samples.dtx  (with options: `all,journal,bibtex,acmtog')
%% 
%% IMPORTANT NOTICE:
%% 
%% For the copyright see the source file.
%% 
%% Any modified versions of this file must be renamed
%% with new filenames distinct from sample-acmtog.tex.
%% 
%% For distribution of the original source see the terms
%% for copying and modification in the file samples.dtx.
%% 
%% This generated file may be distributed as long as the
%% original source files, as listed above, are part of the
%% same distribution. (The sources need not necessarily be
%% in the same archive or directory.)
%%
%%
%% Commands for TeXCount
%TC:macro \cite [option:text,text]
%TC:macro \citep [option:text,text]
%TC:macro \citet [option:text,text]
%TC:envir table 0 1
%TC:envir table* 0 1
%TC:envir tabular [ignore] word
%TC:envir displaymath 0 word
%TC:envir math 0 word
%TC:envir comment 0 0
%%
%% The first command in your LaTeX source must be the \documentclass
%% command.
%%
%% For submission and review of your manuscript please change the
%% command to \documentclass[manuscript, screen, review]{acmart}.
%%
%% When submitting camera ready or to TAPS, please change the command
%% to \documentclass[sigconf]{acmart} or whichever template is required
%% for your publication.
%%
%%
\documentclass[acmtog]{acmart}
%%
%% \BibTeX command to typeset BibTeX logo in the docs
\AtBeginDocument{%
  \providecommand\BibTeX{{%
    Bib\TeX}}}

%% Rights management information.  This information is sent to you
%% when you complete the rights form.  These commands have SAMPLE
%% values in them; it is your responsibility as an author to replace
%% the commands and values with those provided to you when you
%% complete the rights form.
\setcopyright{acmlicensed}
\copyrightyear{2025}
\acmYear{2025}
\acmDOI{XXXXXXX.XXXXXXX}

%%
%% These commands are for a JOURNAL article.
\acmJournal{TOG}
% \acmVolume{37}
% \acmNumber{4}
\acmArticle{1}
\acmMonth{10}

%%
%% Submission ID.
%% Use this when submitting an article to a sponsored event. You'll
%% receive a unique submission ID from the organizers
%% of the event, and this ID should be used as the parameter to this command.
%% \acmSubmissionID{123-A56-BU3}

%%
%% For managing citations, it is recommended to use bibliography
%% files in BibTeX format.
%%
%% You can then either use BibTeX with the ACM-Reference-Format style,
%% or BibLaTeX with the acmnumeric or acmauthoryear sytles, that include
%% support for advanced citation of software artefact from the
%% biblatex-software package, also separately available on CTAN.
%%
%% Look at the sample-*-biblatex.tex files for templates showcasing
%% the biblatex styles.
%%

%%
%% The majority of ACM publications use numbered citations and
%% references.  The command \citestyle{authoryear} switches to the
%% "author year" style.
%%
%% If you are preparing content for an event
%% sponsored by ACM SIGGRAPH, you must use the "author year" style of
%% citations and references.
\citestyle{acmauthoryear}


%%
%% end of the preamble, start of the body of the document source.
\begin{document}

%%
%% The "title" command has an optional parameter,
%% allowing the author to define a "short title" to be used in page headers.
\title{Homework 1 Report}

%%
%% The "author" command and its associated commands are used to define
%% the authors and their affiliations.
%% Of note is the shared affiliation of the first two authors, and the
%% "authornote" and "authornotemark" commands
%% used to denote shared contribution to the research.
\author{Kai Hogan}
% \authornote{Both authors contributed equally to this research.}
\email{khogan@uoregon.edu}



%%
%% By default, the full list of authors will be used in the page
%% headers. Often, this list is too long, and will overlap
%% other information printed in the page headers. This command allows
%% the author to define a more concise list
%% of authors' names for this purpose.
\renewcommand{\shortauthors}{Hogan}

%%
%% The abstract is a short summary of the work to be presented in the
%% article.
\begin{abstract}
This report investigates the performance and accuracy of parallel methods for estimating π using OpenMP. Two approaches were implemented and evaluated: a deterministic numerical integration method and a stochastic Monte Carlo method. Both were compared against a serial baseline to assess accuracy and scalability across varying thread counts. Initial parallel implementations employed critical sections for shared variable updates, which resulted in poor performance. A separate implementation replaced critical regions with atomic operations, yielding measurable improvements at low to moderate thread counts. However, as the number of threads increased beyond eight, performance did not scale due to thread contention and poor scaling. The deterministic integration method consistently produced more accurate results than the Monte Carlo approximation, while the Monte Carlo method executed more quickly. Overall, the study highlights the trade-offs between synchronization mechanisms in OpenMP and the need for reduction-based strategies to achieve efficient parallel performance in embarrassingly parallel numerical computations.
\end{abstract}


%%
%% The code below is generated by the tool at http://dl.acm.org/ccs.cfm.
%% Please copy and paste the code instead of the example below.
%%
\begin{CCSXML}
<ccs2012>
   <concept>
       <concept_id>10010147.10010169.10010170</concept_id>
       <concept_desc>Computing methodologies~Parallel algorithms</concept_desc>
       <concept_significance>500</concept_significance>
       </concept>
   <concept>
       <concept_id>10010147.10010169.10010170.10010171</concept_id>
       <concept_desc>Computing methodologies~Shared memory algorithms</concept_desc>
       <concept_significance>300</concept_significance>
       </concept>
   <concept>
       <concept_id>10003752.10003809</concept_id>
       <concept_desc>Theory of computation~Design and analysis of algorithms</concept_desc>
       <concept_significance>300</concept_significance>
       </concept>
 </ccs2012>
\end{CCSXML}

\ccsdesc[500]{Computing methodologies~Parallel algorithms}
\ccsdesc[300]{Computing methodologies~Shared memory algorithms}
\ccsdesc[300]{Theory of computation~Design and analysis of algorithms}
%%
%% Keywords. The author(s) should pick words that accurately describe
%% the work being presented. Separate the keywords with commas.
% \keywords{Keywords}

% \received{20 February 2007}
% \received[revised]{12 March 2009}
% \received[accepted]{5 June 2009}

%%
%% This command processes the author and affiliation and title
%% information and builds the first part of the formatted document.
\maketitle

\section{Findings}
\subsection{Serial vs Parallel (Deterministic integration): }
For the deterministic integration method, the serial version performed much better than the parallel implementation when using 8 threads with 100 and 1,000 steps. With 10,000 steps, the parallel version showed a slight speedup, though the improvement was not significant. At 100,000 steps, however, the parallel code ran about four times faster than the serial version. For all higher thread counts, the serial implementation consistently outperformed the parallel one at every step size. Still, the performance gap between the two versions became smaller as the number of steps increased.

\begin{figure}[h]
    \centering
    \includegraphics[width=1\linewidth]{graph_1.png}
    \caption{Execution times Deterministic integration (Parallel vs Serial)}
    \label{fig:placeholder}
    \Description{Graph 1}
\end{figure}

\subsection{Parallel(Deterministic integration) vs Parallel(Monte Carlo)}
For smaller step sizes, the Monte Carlo method consistently ran faster than the deterministic integration method. However, as the step size increased, the difference in runtime between the two approaches began to shrink. With 16 and 24 threads at 100,000 steps, both methods produced nearly identical execution times. Despite its speed advantage, the Monte Carlo method was far less accurate than deterministic integration. With only 100 guesses, its estimate of pi could be off by as much as 0.2. Even at higher numbers of guesses, the error remained around 0.05. In contrast, the deterministic integration method was never off by more than 0.01 and became increasingly accurate as the step size grew.

\begin{figure}[h]
    \centering
    \includegraphics[width=1\linewidth]{graph_2.png}
    \caption{Monte Carlo accuracy vs Deterministic integration accuracy}
    \label{fig:placeholder}
    \Description{Graph 2}
\end{figure}

\subsection{Critical Section vs Atomic}
With only 8 threads, the atomic implementation only slightly outperformed the critical section version. As the number of threads increased, the performance gap widened, with the atomic version running significantly faster. Both approaches introduced serialization, but the atomic operation only serialized updates to the pi variable, whereas the critical section serialized access to an entire region of code.

\subsection{Conclusion}
The experiments demonstrate clear trade-offs between parallelization strategies and numerical methods for estimating pi. Deterministic integration consistently produced highly accurate results, while the Monte Carlo method offered faster execution at the cost of lower precision. Overall, achieving both high accuracy and efficient parallel performance requires careful selection of synchronization mechanisms and workload size.

\begin{figure}[h]
    \centering
    \includegraphics[width=1\linewidth]{graph_3.png}
    \caption{Execution times (Critical Section vs Atomic Operation)}
    \label{fig:placeholder}
    \Description{Graph 3}
\end{figure}

%%
%% If your work has an appendix, this is the place to put it.
% \appendix

% \section{Research Methods}

% \subsection{Part One}

% Lorem ipsum dolor sit amet, consectetur adipiscing elit. Morbi
% malesuada, quam in pulvinar varius, metus nunc fermentum urna, id
% sollicitudin purus odio sit amet enim. Aliquam ullamcorper eu ipsum
% vel mollis. Curabitur quis dictum nisl. Phasellus vel semper risus, et
% lacinia dolor. Integer ultricies commodo sem nec semper.

% \subsection{Part Two}

% Etiam commodo feugiat nisl pulvinar pellentesque. Etiam auctor sodales
% ligula, non varius nibh pulvinar semper. Suspendisse nec lectus non
% ipsum convallis congue hendrerit vitae sapien. Donec at laoreet
% eros. Vivamus non purus placerat, scelerisque diam eu, cursus
% ante. Etiam aliquam tortor auctor efficitur mattis.

% \section{Online Resources}

% Nam id fermentum dui. Suspendisse sagittis tortor a nulla mollis, in
% pulvinar ex pretium. Sed interdum orci quis metus euismod, et sagittis
% enim maximus. Vestibulum gravida massa ut felis suscipit
% congue. Quisque mattis elit a risus ultrices commodo venenatis eget
% dui. Etiam sagittis eleifend elementum.

% Nam interdum magna at lectus dignissim, ac dignissim lorem
% rhoncus. Maecenas eu arcu ac neque placerat aliquam. Nunc pulvinar
% massa et mattis lacinia.

\end{document}
\endinput
%%
%% End of file `sample-acmtog.tex'.